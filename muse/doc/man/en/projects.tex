\chapter{Projects}

  \section{Project}

      A \M\ project \index{project} is represented by a folder holding 
      all files of the project. This are mainly the recorded wave files 
      and the the project file \index{project file}.
      The project file contains all information about the project.
      It also contains all midi data if exists.

      Example of a project folder structure:

      \starttyping
      ~/MusE
            projects
                  song1             
                        song1.med
                        rec1.wav
                        rec1.wca
                        rec2.wav
                        rec2.wca
                  song2
                          .             
                          .             
      \stoptyping

      In the above example {\tt song1} is the project folder 
      \index{project folder} and
      {\tt song1.med} is the project file. 
      The {\tt *.wav} files are audio recordings and the {\tt *.wca}
      files contain precomputed data used for fast screen drawing of
      waveforms.

      The path of the standard project folder 
      \index{standard project folder}  {\tt ~/MusE/projects}
      can be configured in the ''Preferences'' menu.
      

   \section{Select a project}

      After \M\ starts, first a project must be choosen or created.

      Normally the last project will be loaded. If you do not like this
      behaviour in the ''Preferences'' menu you can configure
      a standard project \index{standard project} or tell \M\ to
      always ask for a project.

   \section{Templates}

      If you enter the name of a new project in the project selection
      menu then on OK \M\ will present you a list of templates to
      choose from. The template can be a complete project but 
      without any wave data and normally without any midi data.



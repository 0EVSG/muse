\chapter{Midi}
  \section{Midi Routing}
  \section{Midi Setup}
  \subsection{Einfaches Setup}

      In einem einfachen Setup ist jeder Midi Anschluss des Rechners 
      jeweils mit nur einem Midi Gerät verbunden. Außerdem hat jede
      Midi Spur einen eigenen Midi Kanal.
      Dieses Setup ist am flexibelsten und sollte nach Möglichkeit
      immer verwendet werden.

  \subsection{Erweitertes Setup}

      \index{mehrere Midi Spuren für einen Midi Kanal}
      Im erweiterten Setup werden mehrere Midi Spuren auf den gleichen
      Midi Kanal geroutet. Es ist zu beachten, das Midi Controller
      natürlich immer für alle Spuren des Kanals gelten. Controller 
      können nicht individuell für jede Spur eingestellt werden.

  \subsection{Komplexes Setup}
      \index{mehrere Geräte an einem Midi Anschluß}
      Als komplex bezeichnen wir ein Setup, in dem an einem Midi Anschluß
      mehrere Midi Geräte angeschlossen sind.
      Jedes Gerät belegt einen oder mehrere der verfügbaren 16 Midi
      Kanäle. Es ist darauf zu achten, das die vergebenen Kanäle nicht
      überlappen. Dazu müssen die angeschlossenen Midi Geräte so 
      konfiguriert werden, das sie nur auf die ihnen zugewiesenen Kanäle
      reagieren.

      Midi Meldungen, die nicht kanalgebunden sind (wie z.B. Sysex 
      Meldungen) können an bestimmte Geräte gesendet werden, indem
      man jedem Gerät eine individuelle Geräte-Id gibt. Diese Geräte-Id
      muß dann in die entsprechenden Meldungen als Zieladresse angegeben
      werden. 

      Da Midi Verbindungen nur eine sehr bescheidene \index{Midi Bandbreite}
      Bandbreite besitzen, sollte nach Möglichkeit das komplexe Setup 
      vermieden werden.
      
  \section{Midi Eingänge}
      Ein Midi Eingang wird im Mixer als Strip und im Arranger als
      eigene Spur gezeigt.

      Ein Midi Eingang hat folgende Eigenschaften:

      \blank[big]
      \Input{Alsa Port:} dies ist die Route zu einem ALSA Midi Eingangs
         Port. Ein ALSA Port kann mit mehreren Midi Eingängen verbunden
         werden.
      \Input{Plugins:} die von ALSA gelesenen Midi Events können durch
         mehrere Plugins geroutet werden. Verfügbare Plugins filtern oder
         verändern die Midi Events.
      \Input{Outputs:} ein Midi Eingang hat 16 Ausgänge, jeweils ein
         Ausgang für jeden Midi Kanal. Jeder Midi Kanal kann individuell
         zu einem oder mehreren Midi Spuren geroutet werden.
         Midi Events, die von den Midi Spuren empfangen werden, haben
         keine Kanal-Information mehr.
         In einem einfachen Setup werden alle Kanäle aller Midi Eingänge
         zu allen Midi Spuren geroutet. Dies ist die Standard Vorgabe.

      \blank[big]

  \section{Midi Spur}
      Die Midi Spur enthält alle Midi Note On/off Events. Sie wird im
      Mixer als Strip und im Arranger als Spur dargestellt. Die
      Strip Darstellung im Mixer ist nicht sehr interessant, da eine
      Midi Spur kaum eigene Mixer relevante Parameter besitzt. In einem 
      normalen Setup sollten Midi Spuren im Mixer ausgeblendet werden.

      Midi Spur Eigenschaften:

      \blank[big]
      \Input{Record:} schaltet die Spur in den Record Modus

      \Input{Monitor:} ist der Monitor Schalter eingeschaltet, werden
            bei Record alle Input Events zum Ausgang weitergereicht.
            Achtung: dies kann möglicherweise zu Midi Rückkopplungen
            führen!

      \Input{Mute:} schaltet die Spur stumm

      \Input{Solo:} ---nocht nicht definiert---

      \Input{Eingang:} die Spur Eingänge können zu Midi Eingängen
            geroutet werden

      \Input{Ausgang:} der Spur Ausgang kann zu einem oder mehreren
            Midi Kanälen geroutet werden. Midi Kanäle sind immer mit
            einem Midi Port verbunden.

      \blank[big]

  \section{Midi Kanäle}
      Midi Kanäle enthalten normalerweise alle Midi Controller Events.

      Midi Kanal Eigenschaften:

      \blank[big]

      \blank[big]

  \section{Midi Port}

      \index{Midi Port} Ein Midi Port representiert ein externes Midi 
      Ger� und hat folgende Eigenschaften:

      \blank[big]
      \Input{Instrument:} \index{Midi Instrument} beschreibt die 
        Eigenschaften des externen
        Midi Instruments. Das Instrument kann aus einer Liste verfügbarer
        Instrumentbeschreibungen ausgewählt werden.

      \Input{Master Volume:} \index{Midi Master Volume} ist ein Midi 
      Controller, der die Lautstärke für alle Channel des Midi Ports 
      einstellt.

      \Input{Geräte Id:} \index{Midi Geräte Id} werden mehrere Midigeräte 
      an einen Midi Port
      angeschlossen, so müssen alle Geräte eine eindeutige Geräte Id
      besitzen. Mit dieser Id kann dann ein bestimmtes Gerät
      ausgewählt werden.

      \Input{Alsa Port:} dies ist die Route zu einem ALSA Midi Port.
      Ein Port Ausgang kann mit mehreren ALSA Ports verbunden werden.

      \blank[big]

      Jeder Midi Port besitzt 16 Kanäle.

  \section{Midi Synthesizer}


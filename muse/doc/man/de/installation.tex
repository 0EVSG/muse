%% \chapter{Installation}
  \section{Tips für Ubuntu/Kubuntu}

    \subsection{Alsa-Sequencer}

      \M\ benötigt das ALSA Sequencer Modul. Dies wird von Ubuntu leider
      beim Systemstart nicht installiert.

      Wir können das aber leicht durch einen Eintrag in {\tt /etc/modules}
      beheben:

      \starttyping
            sudo bash
            modprobe snd-seq
            echo snd-seq >> /etc/modules
      \stoptyping

      (Nachtrag: im aktuellen Ubuntu 04.07 ("festy") wird das Sequencer
      Modul standardmäßig geladen)

    \subsection{Realtime Clock}

      Normalen Programmen ist der Zugriff auf die Echtzeituhr
      (RTC = "realtime clock") von Linux
      verwehrt. \M\ benötigt jedoch eine genaue Uhr, um ein exaktes
      Miditiming erzeugen zu können.
      Zunächst wollen wir daher die RTC für alle Audioprogrammen
      verfügbar machen:

      \starttyping
            sudo chown -vv root:audio /dev/rtc
            sudo chmod g+rw /dev/rtc
      \stoptyping

      Jetzt haben alle Programme der audio Gruppe Zugriff auf die
      RTC, dürfen jedoch noch nicht die hohen von \M\ benötigten
      Auflösungen einstellen.
      Um dies zu Erlauben geben wir ein:

      \starttyping
            sudo bash
            echo 1024 > /proc/sys/dev/rtc/max-user-freq
            echo dev.rtc.max-user-freq=1024 >> /etc/sysctl.conf
      \stoptyping

      Durch den Eintrag in der Datei {\tt /etc/sysctl.conf} erfolgt diese
      Einstellung nun auch nach jedem Systemstart.

    \subsection{Realtime Rechte}

      Um Prozesse mit Echtzeit Privilegien starten zu können ergänzen
      wir die Datei {\tt /etc/security/limits.conf} um folgende Zeilen:

      \starttyping
      @audio - rtprio 99
      @audio - memlock 200000
      @audio - nice -10
      \stoptyping

    \subsection{Low Latency Kernel}
      \subsubsection{Interrupt Thread Prioritäten}



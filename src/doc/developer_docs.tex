%% (c) 2012 florian jung
%% (c) 2012 Robert Jonsson
%% we should consider putting this under a proper license. GPL, or
%% some GPL-like documentation license??

%% rules for editing documentation: (READ THIS FIRST)
%% ~~~~~~~~~~~~~~~~~~~~~~~~~~~~~~~~~~~~~~~~~~~~~~~~~~
%%
%% please try to let newly written lines be shorter than 72 characters.
%% minor exceptions are okay, but please not more than 80 chars.
%% comments shall start after character #80 of the line (that is,
%% they shall be "on the right margin")
%%
%% DON'T MIX up changes and reformatting in one commit. when changing
%% stuff, please don't touch the lines before and after your change
%% (that is, do not re-wrap them), even if it will look a bit patchy.
%% this is for being able to easily use diff.
%% when you want to reformat this file, then do it. but don't change
%% anything, as this would be hard to find in a diff. and clearly
%% state in the commit log that you "only" rearranged things.
%%
%% please adhere to the "User's manual" / "Internals" / "Design"
%% partitioning (generally, don't change the chapters until there
%% is a really good reason for doing so (adding a chapter like
%% "feature requests" as flo did in r1497 IS one).
%% Below that, feel free to change the logical arrangement
%% (making paragraphs to subsections and similar) if you deem it
%% necessary.
%%
%% Whenever referring to code symbols, constants or source/header
%% files, please use \sym{someClass::someSymbol}, \usym{UPPERCASE_THING}
%% or \f{file.cpp}.
%% Only specify file paths or namespaces where it would be ambiguous
%% otherwise. Specify 'someClass::' until it would disturb the reader
%% and it is obvious. you have to replace '_' by {\_} (with the {}!).
%% These macros do automatic hyphenation on Camel-Case, under_-score
%% and scope::-operator boundaries. If you need to insert additional
%% hyphenation points, use {\-}.
%% Example: \sym{someClass::someAb{\-}nor{\-}mal{\-}lyLongName} will
%%          hyphenate: some-Class::-some-Ab-nor-mal-ly-Long-Name
%%
%% Whenever referring to URLs, please wrap them in \url{blah}. Key
%% combinations shall look like \key{CTRL+C}. Menu items shall look
%% like \menu{Menu > Submenu > Menu Item}.
%%
%% Where possible, reference other parts of this documents with
%% \label and \ref. Avoid duplicate information under "Internals" by
%% referring to the appropriate section in "User's manual".
%%
%% Please do no time-stamping of sections. if you need time-stamps,
%% use "svn blame documentation.tex"
%%
%% If you contribute something, feel free to add yourself to \author.
%%
%% If you don't speak LaTeX fluently, a few tips:
%% * \section, \subsection, \subsubsection, \paragraph, \subparagraph
%%   let you create sections etc. just copy-and-paste if unsure.
%% * you must prefix special characters like the underscore with \
%%   (backslash)
%% * \emph{some text} emphasizes the text, printing it italic.
%% * \texttt{some text} displays the text in a typewriter font
%% * \label{someName} creates a label at this position. this doesn't
%%   show up in the pdf. with \ref{someName}, you can reference to this
%%   label. (LaTeX will insert the section number instead of \ref)
%%   For this to work, you might need to recompile the .tex twice.
%%
%%
%%%%%%%%%%%%%%%%%%%%%%%%%%%%%%%%%%%%%%%%%%%%%%%%%%%%%%%%%%%%%%%%%%%%%%%%
%%
%% Dependencies:
%% To produce pdf output from this document the following packages are
%% needed (atleast under Ubuntu derivatives)
%% latex???-base
%% texlive-latex-recommended
%% texlive-latex-extra
%%
%% To produce a pdf version of this manual type:
%% pdflatex documentation.tex <enter>
%% A file documentation.pdf should be generated.
%%
%%%%%%%%%%%%%%%%%%%%%%%%%%%%%%%%%%%%%%%%%%%%%%%%%%%%%%%%%%%%%%%%%%%%%%%%



\documentclass[a4paper]{report}
\usepackage[a4paper,  left=2.5cm, right=2.5cm, top=2.5cm, bottom=2.5cm]{geometry}
\usepackage[T1]{fontenc}
\usepackage[utf8]{inputenc}
\usepackage{lmodern}
\usepackage[english]{babel}
\usepackage{graphicx}
\usepackage{hyphenat}
\usepackage{wrapfig}
\usepackage{fancyhdr}
\usepackage{hyperref}
\usepackage{xcolor}
\hypersetup{
  linkcolor=blue,
  urlcolor=blue,
  colorlinks=true,         % hyperlinks will be blue
  linkbordercolor=blue,     % hyperlink borders will be blue
  %pdfborderstyle={/S/U/W 1} % border style will be underline of width 1pt
}
\pagestyle{fancy}
        \lhead{\scriptsize{\slshape\leftmark}}
        \chead{}
        \rhead{\thepage}
        \lfoot{}
        \cfoot{}
        \rfoot{}
        \renewcommand{\headrulewidth}{0.4pt}
\usepackage{ifthen}

%% Borrowed from Lyx output. "Because html converters don't know
%%  tabularnewline". Is required? Maybe replace with \\ ?
\providecommand{\tabularnewline}{\\}

% Hyphenate symbols before each uppercase letter, after each underscore
% (without a "-") and after each ':' (without a "-")
% TODO for any latex crack: also do automatic hyphenation, that is,
% instead of some-Automation-Expression, do some-Au-to-ma-tion-Ex-press-ion
\makeatletter
\newcommand{\camelhyph}[1]{\@fterfirst\c@amelhyph#1\relax }
\newcommand{\underscorehyph}[1]{\@fterfirst\u@nderscorehyph#1\relax }
\def\@fterfirst #1#2{#2#1}
\def\c@amelhyph #1{%
\ifthenelse{\equal{#1}\relax}{}{%  Do nothing if the end has been reached
  \ifnum`#1=95 \_\hspace{0pt}\else      %     Check whether #1 is "_", then print _[thin space]
    \ifnum`#1=58 :\hspace{0pt}\else
      \ifnum`#1>64
        \ifnum`#1<91 \-#1\else#1\fi%     Check whether #1 is an uppercase letter,
      \else#1\fi
    \fi
  \fi
                            %     if so, print \-#1, otherwise #1
  \expandafter\c@amelhyph%    %     insert \c@amelhyph again.
}}
\def\u@nderscorehyph #1{%
\ifthenelse{\equal{#1}\relax}{}{%  Do nothing if the end has been reached
  \ifnum`#1=95 \_\hspace{0pt}\else      %     Check whether #1 is "_", then print _\-
    \ifnum`#1=58 :\hspace{0pt}\else#1\fi\fi
  \expandafter\u@nderscorehyph%    %     insert \u@nderscorehyph again.
}}
\makeatother



\author{Florian Jung, Robert Jonsson, Tim Donnelly}
\title{MusE Documentation}

% Orcan: not needed since the hyperref package takes care of all
%\newcommand{\url}[1]{\texttt{#1}}
\newcommand{\key}[1]{\textbf{#1}}
\newcommand{\shell}[1]{\texttt{\textbf{#1}}}
\newcommand{\menu}[1]{\textbf{#1}}
\newcommand{\sym}[1]{\texttt{\camelhyph{#1}}}
\newcommand{\listing}[1]{\texttt{\camelhyph{#1}}}
\newcommand{\usym}[1]{\texttt{\underscorehyph{#1}}}
\newcommand{\file}[1]{\texttt{\camelhyph{#1}}}
\newcommand{\screenshotwidth}[0]{0.8\textwidth}

\begin{document}
\tableofcontents
\chapter{Internals -- how it works}
This chapter explains how MusE is built internally, and is meant
to be an aid for developers wanting to quickly start up with MusE.
For details on \emph{why} stuff is done please refer to the following
chapter.

\section{User interface programming}
We use the QT Toolkit for GUI- and other programming. The \emph{QT-Assistant}
is an important tool for getting help. Almost everything can be looked
up there.

GUIs can be either be hardcoded (see \file{arranger.cpp} for an example)
or can be created using \emph{QT-Designer} (see the dialogs under
\file{widgets/function_dialogs/} for mostly cleanly-written examples).
Don't forget to add your \file{cpp}, \file{h} and \file{ui} files to the
corresponding sections in the \file{CMakeLists.txt}!

Additionally, MusE offers some custom widgets, like menu title items etc.
Following, there will be a small, unordered list about custom widgets:
\begin{itemize}
\item \sym{MusEGui::MenuTitleItem}: Provides a title-bar in a \sym{QMenu}. \\
      Usage: \listing{someMenu->addAction(new\ MusEGui::MenuTitleItem(tr("fnord"),\ someMenu));} \\
      Defined in \file{widgets/menutitleitem.h}.
\item \sym{MusEGui::PopupMenu}: Provides a \sym{QMenu}-like menu which
      can stay open after the user checks a checkable action.\\
      Usage: just create a \listing{new\ PopupMenu(\ true|false\ )} instead of
             a \listing{new\ QMenu()}. (\listing{true} means 'stay open')\\
      Defined in \file{widgets/popupmenu.h}.
\end{itemize}


\section{Configuration} \label{portconfig_sucks}
Configuration is a bit pesky in MusE in its current state. If you get
confused by reading this chapter, that's a sign of a sane mind.

There are three kinds of configuration items:
\begin{itemize}
\item (1) Global configuration, like coloring schemes, plugin categories, MDI-ness settings
\item (2) Per-Song configuration, like whether to show or hide certain track types in the arranger
\item (3) Something in between, like MIDI port settings etc. They obviously actually are
      global configuration issues (or ought to be), but also obviously must be stored
      in the song file for portability. (This problem could possibly be solved by
      the feature proposal in \ref{symbolic_ports}.
\end{itemize}

\paragraph{Reading configuration}
\sym{void\ MusECore::readConfiguration(Xml\&,\ bool,\ bool)} in
\file{conf.cpp} is the central point
of reading configuration. It is called when MusE is started first
(by \sym{bool\ MusECore::readConfiguration()}), and also when a
song is loaded. \\ %TODO: call paths!
It can be instructed whether to read MIDI ports (3), global configuration
and MIDI ports (1+3). Per-Song configuration is always read (2).

When adding new configuration items and thus altering \sym{readConfiguration()},
you must take care to place your item into the correct section. The code is
divided into the following sections:
\begin{itemize}
\item Global and/or per-song configuration (3)
\item Global configuration (1)
\item Code for skipping obsolete entries
\end{itemize}

The sections are divided by comments (they contain \texttt{----}, so just
search for them). Please do not just remove code for reading obsolete entries,
but always add an appropriate entry to the 'skipping' section in order to
prevent error messages when reading old configs.

\paragraph{Writing configuration}
Global configuration is written using the
\sym{MusEGui::MusE::writeGlobalConfiguration()} functions, while
per-song-config is written by \sym{MusEGui::MusE::writeConfiguration()}
(notice the missing \sym{Global}; both implemented in \file{conf.cpp}).

\sym{writeConfiguration} is actually just a subset of the code in
\sym{writeGlobalConfiguration}. \textbf{Duplicate code!}                  % TODO fix that in the sourcecode.

\paragraph{Song state}
Additionally to per-song configuration, there is the song's state.
This contains "the song", that is all tracks, parts and note events,
together with information about the currently opened windows, their
position, size, settings and so on. Adding new items here is actually
pretty painless: Configuration is read and written using
\sym{MusECore::Song::read} and \sym{::write}, both implemented in
\file{songfile.cpp}. There are no caveats.

\paragraph{How to add new items}
When adding global configuration items, then add them into the second
block ("global configuration") in \sym{readConfiguration} and into
\sym{writeGlobalConfiguration}.

When adding just-per-song items, better don't bother to touch the
"configuration" code and just add it to the song's state (there might
be rare exceptions).

When adding global configuration items, make sure you add them into the
correct section of \sym{readConfiguration}, and into \sym{writeGlobalConfiguration}.


%TODO: how to write config,
%TODO: song/global stuff
%TODO: config lesen und schreiben fuer plugingroups


\section{User controls and automation}
\subsection{Handling user input}
\subsubsection{Plugins and synthesizers}
\paragraph{Overview}
When the user launches a plugin's GUI, either a MusE-window with
the relevant controls is shown, or the native GUI is launched. MusE
will communicate with this native GUI through OSC (Open Sound Control).
The relevant classes are \sym{PluginGui}, \sym{PluginIBase}
(in \file{plugin.h}) and \sym{OscIF} (in \sym{osc.h}).

If the user changes a GUI element, first the corresponding control is
disabled, making MusE not steadily update it through automation
while the user operates it. Then MusE will update the plugin's parameter
value, and also record the new value. When appropriate, the controller
is enabled again.

\paragraph{Processing the input, recording}
Upon operating a slider, \sym{PluginIBase::setParam} is called,
which usually writes the control change into the ringbuffer
\sym{PluginI::{\_}controlFifo}. (\sym{PluginI::apply()},
\sym{DssiSynthIF::getData()} will read this ringbuffer and
do the processing accordingly). Furthermore, \sym{AudioTrack::recordAutomation}
is called, which either directly modifies the controller lists or writes
the change into a "to be recorded"-list (\sym{AudioTrack::{\_}recEvents})
(depending on whether the song is stopped or played).

The \sym{AudioTrack::{\_}recEvents} list consists of \sym{CtrlRecVal}
items (see \file{ctrl.h}), which hold the following data:
\begin{itemize}
\item the frame where the change occurred
\item the value
\item the type, which can be \usym{ARVT{\_}START}, \usym{ARVT{\_}VAL} or \usym{ARVT{\_}STOP}.
      \usym{ARVT{\_}VAL} are written by every \usym{AudioTrack::recordAutomation}
      call, \usym{ARVT{\_}START} and \usym{ARVT{\_}STOP} are generated by
      \sym{AudioTrack::startAutoRecord} and \sym{stopAutoRecord}, respectively.
\item and the id of the controller which is affected
\end{itemize}
It is processed when the song is stopped. The call path for this is:
\sym{Song::stopRolling} calls \sym{Song::processAutomationEvents}
calls \sym{AudioTrack::processAutomationEvents}.
This function removes the old events from the track's controller list
and replaces them with the new events from \sym{{\_}recEvents}. In
\usym{AUTO{\_}WRITE} mode, just all controller events within the recorded
range are erased; in \usym{AUTO{\_}TOUCH} mode, the \usym{ARVT{\_}START}
and \usym{ARVT{\_}STOP} types of the \sym{CtrlRecVal} events are used
to determine the range(s) which should be wiped.

\paragraph{How it's stored}
Automation data is kept                                                  % this is copied from
in \sym{AudioTrack::{\_}controller}, which is a \sym{CtrlListList},        % "design decisions" -> "automation"
that is, a list of \sym{CtrlList}s, that is, a list of lists of
controller-objects which hold the control points of the automation graph.
The \sym{CtrlList} also stores whether the list is meant discrete
(a new control point results in a value-jump) or continuous (a new control
point results in the value slowly sloping to the new value).
Furthermore, it stores a \sym{{\_}curVal} (accessed by \sym{curVal()}),
which holds the currently active value, which can be different from the
actually stored value because of user interaction. This value is also
used when there is no stored automation data.

\sym{AudioTrack::addController} and \sym{removeController} are used        % TODO: swapControllerIDX??
to add/remove whole controller types; the most important functions which
access \sym{{\_}controller} are:
\begin{itemize}
\item \sym{processAutomationEvents}, \sym{recordAutomation},
      \sym{startAutoRecord}, \sym{stopAutoRecord}: see above.
\item \sym{seekPrevACEvent}, \sym{seekNextACEvent}, \sym{eraseACEvent},
      \sym{eraseRangeACEvents}, \sym{addACEvent}, \sym{changeACEvent},
      which do the obvious
\item \sym{pluginCtrlVal}, \sym{setPluginCtrlVal}: the first
      returns the current value according to the \sym{{\_}controller}
      list, the second only sets the \sym{curVal}, but does not
      insert any events.
\end{itemize}

Whenever a \sym{CtrlList} has been manipulated,
\sym{MusEGlobal::song->controllerChange(Track{*})} shall be called,
which emits the \sym{MusEGlobal::song->controllerChanged(Track{*})}
signal in order to inform any parts of MusE about the change (currently,
only the arranger's part canvas utilizes this).

\paragraph{Enabling and disabling controllers}
Disabling the controller is both dependent from the current automation
mode and from whether the GUI is native or not.
In \usym{AUTO{\_}WRITE} mode, once a slider is touched (for MusE-GUIs) or
once a OSC control change is received (for native GUIs), the control
is disabled until the song is stopped or seeked.

In \usym{AUTO{\_}TOUCH} (and currently (r1492) \usym{AUTO{\_}READ}, but
that's to be fixed) mode, once a MusE-GUI's slider is pressed down, the
corresponding control is disabled. Once the slider is released, the
control is re-enabled again. Checkboxes remain in "disabled" mode,
however they only affect the recorded automation until the last toggle
of the checkbox. (Example: start the song, toggle the checkbox, toggle
it again, wait 10 seconds, stop the song. This will NOT overwrite the
last 10 seconds of automation data, but everything between the first
and the last toggle.). For native GUIs, this is a bit tricky, because
we don't have direct access to the GUI widgets. That is, we have no
way to find out whether the user doesn't touch a control at all, or
whether he has it held down, but just doesn't operate it. The current
behaviour for native GUIs is to behave like in \usym{AUTO{\_}WRITE} mode.

The responsible functions are: \sym{PluginI::oscControl} and
\sym{DssiSynthIF::oscControl} for handling native GUIs,
\sym{PluginI::ctrlPressed} and \sym{ctrlReleased} for MusE
default GUIs and \sym{PluginI::guiParamPressed},
\sym{guiParamReleased}, \sym{guiSliderPressed} and
\sym{guiSliderReleased} for MusE GUIs read from a UI file;
\sym{guiSlider{*}} obviously handle sliders, while \sym{guiParam{*}}
handle everything else which is not a slider. They call
\sym{PluginI::enableController} to enable/disable it.

Furthermore, on every song stop or seek, \sym{PluginI::enableAllControllers}
is called, which re-enables all controllers again. The call paths for
this are:
\begin{itemize}
\item For stop: \sym{Song::stopRolling} calls
                \sym{Song::processAutomationEvents} calls
                \sym{Song::clearRecAu{\-}to{\-}ma{\-}tion} calls
                \sym{Track::clearRecAutomation} calls
                \sym{PluginI::enableAllControllers}
\item For seek: \sym{Audio::seek} sends a message ("\sym{G}") to
                \sym{Song::seqSignal} which calls
                \sym{Song::clearRecAutomation} which calls
                \sym{PluginI::enableAllControllers}
\end{itemize}




\chapter{Design decisions}
\section{Automation}
As of revision 1490, automation is handled in two ways: User-generated
(live) automation data (generated by the user moving sliders while playing)
is fed into \sym{PluginI::{\_}controlFifo}. Automation data is kept
in \sym{AudioTrack::{\_}controller}, which is a \sym{CtrlListList},
that is, a list of \sym{CtrlList}s, that is, a list of lists of
controller-objects which hold the control points of the automation graph.
The \sym{CtrlList} also stores whether the list is meant discrete
(a new control point results in a value-jump) or continuous (a new control
point results in the value slowly sloping to the new value).

While \sym{PluginI::{\_}controlFifo} can be queried very quickly and
thus is processed with a very high resolution (only limited by the
minimum control period setting), the automation value are expensive to
query, and are only processed once in an audio \emph{driver} period.
This might lead to noticeable jumps in value.

This could possibly be solved in two ways:
\paragraph{Maintaining a slave control list}
This approach would maintain a fully redundant slave control list,
similar to \sym{PluginI::{\_}controlFifo}. This list must be updated
every time any automation-related thing is changed, and shall contain
every controller change as a tuple of controller number and value.
This could be processed in the same loop as \sym{PluginI::{\_}controlFifo},
making it comfortable to implement; furthermore, it allows to cleanly
offer automation-settings at other places in future (such as storing
automation data in parts or similar).

\paragraph{Holding iterators}
We also could hold a list of iterators of the single \sym{CtrlList}s.
This would also cause low CPU usage, because usually, the iterators only
need to be incremented once. However, it is pretty complex to implement,
because the iterators may become totally wrong (because of a seek in the
song), and we must iterate through a whole list of iterators.

\paragraph{Just use the current data access functions}
By just using the current functions for accessing automation data,
we might get a quick-and-dirty solution, which however wastes way too
much CPU resources. This is because on \emph{every single frame}, we
need to do a binary search on multiple controller lists.


\chapter{Feature requests}
\section{Per-Part automation and more on automation}                       % by flo
Automation shall be undo-able. Automation shall reside in parts which
are exchangeable, clonable etc (like the MIDI- and Wave-Parts).
Global per-synth/per-audiotrack automation shall also be available, but
this can also be implemented as special case of part automation (one
long part).

\section{Pre-Rendering tracks}
\subsection{The feature}
All tracks shall be able to be "pre-renderable". Pre-rendering shall
be "layered". Pre-rendering shall act like a transparent audio cache:
Audio data is (redundantly) stored, wasting memory in order to save CPU.

That is: Each track owns one or more wave-recordings of the length of
the song. If the user calls "pre-render" on a track, then this track
is played quasi-solo (see below), and the raw audio data is recorded
and stored in the "layer 0" wave recording. If the user has any effects
set up to be applied, then each effect is applied on a different layer
(creating layer 1, layer 2 etc).

This means, that also MIDI and drum tracks can have effects (which
usually only operate on audio, but we HAVE audio data because of this
prerendering).

Furthermore, MusE by default does not send MIDI events to the synthesizers
but instead just plays back the last layer of the prerecording (for
MIDI tracks), or does not pipe the audio data through the whole plugin
chain (causing cpu usage), but instead just plays back the last layer.
The hearable result shall be the same.

Once the user changes any parameter (automation data or plugins for
wave tracks, MIDI events or effect plugin stuff for MIDI tracks),
then MusE shall generate the sound for this particular track in the
"old" way (send MIDI data to synthes, or pipe audio data through plugins).
(So that the user will not even notice that MusE actually pre-renderered
stuff.) Either MusE automatically records this while playback (if possible)
or prompts the user to accordingly set up his cabling and then record
it. Or (temporarily) disables prerecording for this track, falling back
to the plain old way of generating sound.

\emph{Quasi-solo} means: For wave tracks, just solo the track. For MIDI
tracks, mute all tracks which are not on the same synth (channel?),
and mute all \emph{note} events which are not on the quasi-soloed track.
This causes MusE to still play any controller events from different
tracks, because they might have effects on the quasi-soloed track. (You
can have notes on channel 1 on one track and controller stuff on channel
1 on another track; then you would need quasi-solo to get proper results.)

\subsection{Use cases}
\paragraph{Saving CPU}
On slow systems, this is necessary for songs with lots of, or demanding
(or both) soft synthes / plugins. Even if the synth or plugin is so
demanding that your system is not able to produce sound in real-time,
then with this feature you'll be able to use the synth (this will make
editing pretty laggish, because for a change you need to re-render at
least a part before you can listen to it, but better than being unable
to use the synth at all!)

\paragraph{Exporting as audio project}
Using pre-rendering on all tracks, you easily can export your project
as multi-track audio file (for use with Ardour or similar DAWs).
Just take the last layer of each track, and write the raw audio data
into the file, and you're done. (Maybe we are even able to write down
the raw-raw layer0 audio data plus information about used plugins and
settings etc..?)

\paragraph{Mobile audio workstations}
You might want to work a bit on your audio projects on your notebook
while you're not at home, not having access to your hardware synthesizers.
Using this feature, you could have pre-recorded the stuff in your studio
before, and now can at least fiddle around with the non-hw-synth-dependent
parts of your song, while still having your \emph{full} song with you.

\paragraph{Applying effects on MIDI tracks}
If you have many physical audio inputs, you might already be able to
apply effect chains on MIDI tracks, by wiring the synthes' audio
outputs to your soundcard's inputs, and applying the effects on
dedicated input tracks you have to create. This requires you to have
expensive hardware, and is pretty complicated, because you need one
additional track per MIDI synth.

This feature allows you to apply effects on single MIDI tracks, and not
only on full MIDI synthes, and doesn't require you to be have that
many physical audio inputs (you need to manually replug your synthes,
however).

\subsection{Possible scenarios}
\paragraph{Setting it up}
Create a wave track, MusE will allow you to set or unset prerendering
for every plugin in the plugin rack (recording the actual track is
useless because it would be a plain copy).
Create a MIDI track, MusE will ask you on which physical audio input
your synth is connected. Setting up multiple synthes on one physical
audio in is allowed, see below.

\paragraph{Pre-rendering stuff}
When the user presses the "pre-render" button, all tracks which have
been changed since their last pre-rendering will be re-rendered.
If you have multiple hardware synthes set up as they were connected
to one physical audio input port, MusE will prompt you to first plug
the proper cable in.

\paragraph{Making changes}
Change a note in a MIDI part, move or delete a part or change automation
parameters. MusE will temporarily disable the pre-rendered information
and instead generate the sound via sending out MIDI events, piping stuff
through effect chains or similar. If you play back the whole song, or
if you manually trigger a re-rendering of a track via the context menu,
MusE will play back the stuff, record it again and re-enable the
pre-rendered information.


\subsection{Extensions}
\paragraph{Automatic discovery of physical audio connections}
The user plugs all (or only some) synthes' audio outs into the available
audio inputs, then runs automatic discovery. This will send MIDI events
to each synthesizer, and look on which audio in there's activity. Then
it will assume that the synthesizer is connected to that particular
audio in. Audio ins which show activity before any MIDI events were
sent are not considered, as they're probably connected to microphones
or other noise-generating non-synthes.

\paragraph{Audio export}
As described in the Use cases, MusE can allow you to export your song
in some multitrack audio format.

\paragraph{Cheap/Faked changes}
For expensive or unavailable synthes, changing the Volume midi controller,
the Pan controller or similar "easy" controllers will not trigger a
complete re-rendering, but instead "fake" the change, by changing
the volume data directly on the recorded wave. This might require some
learning and might even get pretty complicated.

\paragraph{Intelligent re-recording}
For tiny changes, MusE shall only re-render the relevant part. If you
change some MIDI notes, then begin re-recording shortly before the
changes, and end re-recording as soon as the recorded stuff doesn't
differ to much from the stuff coming from the synth. Then properly
blend the old recording with the updated part.



\section{Slotted editors}
Currently, MusE has the pianoroll editor, drum editor, score editor,
then the controller editor which is inside the pianoroll/drum editor.
All these editors have a very similar concept: the "time axis" is
vertical and (almost) linear, they handle parts, and events are
manipulated similarly.

A unified editor shall be created which allows you to combine different
kinds of editors in one window, properly aligned against each other.
These "different kinds of editors" shall be handled as "slots"; one
unified editor window consists of:
\begin{itemize}
\item A menu bar, containing stuff suitable for the complete window,
      which might include window name, MDI-ness etc.
\item A toolbar which contains controls suitable for every single slot.
\item A container with one or more slots; the slots can be scrolled in
      y-direction if there are multiple slots.
\item A time-scrollbar with zoom
\end{itemize}

Each slot contains the following:
\begin{itemize}
\item A menu button, button box or control panel for setting up this
      particular slot. This could contain "note head colors", "show
      a transposing instrument" etc for score edit slots, "event
      rectangle color", "grid size" and "snap to grid" for pianoroll/
      drum editors.
\item The actual canvas
\item A y-direction scroll bar, possibly with zoom control (for
      pianoroll editor)
\end{itemize}

The main window does not show its scroll bar if there is only one slot,
because the slot's scrollbar is sufficient then.

Slots can be added, destroyed, moved around, maybe even merged (if the
slot types allow it); basically, you can compare them with the staves
in the score editor.

The slots shall align against each other, that is, if a score editor
slot displays a key change with lots of accidentials, then all other
slots shall either also display the key change (if they're score slots)
or display a gap. Events which happen at the same time shall be at the
same x-coordinate, regardless which slot they are.

\section{Controller master values}
All controllers (MIDI-controllers and also automation controllers)
shall have one set of "master values" which allow you to set a gain and
a bias. Instead of the actual set value, $\textrm{value} * \textrm{bias}
+ textrm{bias}$ shall be sent to the MIDI device / the plugin. For
controllers like "pan", the unbiased values shall be transformed, that
is, a pan of 64, with $\textrm{bias}=2$ and $\textrm{gain}=0.5$, shall
be transformed to 66 (because 64 is actually 0, while 0 is actually -64).
These values shall be set in the arranger and wherever the actual
controller/automation values can be edited.

\section{Enabled-indicator while recording}
The MusE-plugin-GUIs shall display a small LED displaying whether a
controller is currently enabled or disabled. By clicking this LED, the
enabled state shall be switched.

Furthermore, there shall be a dedicated window which only lets you switch
enabled/disabled states. This will be useful when using external GUIs
or the MIDI-controller-to-automation feature, to re-enable a controller
when in \usym{AUTO{\_}TOUCH} mode.



\section{Linear automation editing}
While holding some modifier key (like shift), operating the MusE-native-
GUI sliders shall only generate control points when clicking and when
releasing the slider. This will result in linear graphs for continuous
controllers, and in large steps for discrete controllers (which is in
particular useful for stuff like "which low/high-pass filter type to use").

Maybe make this behaviour default for discrete controllers?

\section{Symbolic names for MIDI ports} \label{symbolic_ports}
MIDI ports shall have a user-defined symbolic name (like "Korg" or "Yamaha DX 7").
The mapping between these symbolic names and the hardware port (like
"ALSA midi out port") is stored in the global configuration.

Song files only specify the symbolic names as the ports associated with
their tracks. No information about physical devices/port names, but only
symbolic names are stored in the song file.

This resolves the issues mentioned in \ref{portconfig_sucks}, and also
allows the user to share his pieces with other people: They would only
have to set up that symbolic-to-hardware mapping once (collisions are
unlikely, because an equal symbolic name should usually mean the same
device) and are happy, instead of having to re-map \emph{every} port
for \emph{every} song.

\end{document}

% TODO: song type etc? kill it!
%       song len box: same

% TODO: unified automation and midi ctrls:
%       both shall be editable through the same editors
%       four modes: 1. discrete
%                   2. continuous (plus a global and per-port setting for the max rate)
%                   3. switch (not necessarily ON/OFF; signals with color plus text annotation)
%                   4. raw (no graph, instead a box with the value sent out (for "all voices off")
%       they shall be copy-and-pastable, at least between compatible modes
%       they shall be slotted, like pianoroll
%       maybe also "overlapping", like arranger (?)
%       midi recording tries to make out straight lines (like non-ended automation streams)
%


% new song (in general: load as template) plus "overwrite port config"
% should re-create the default jack devices and autoconnect them.

% what's audio aux for?

% bug in arranger/pcanvas/automation: if a controlpoint is directly on
% a line of another ctrl graph, you can't click it

